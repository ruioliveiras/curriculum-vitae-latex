%%%%%%%%%%%%%%%%%%%%%%%%%%%%%%%%%%%%%%%%%
% Friggeri Resume/CV
% XeLaTeX Template
% Version 1.0 (5/5/13)
%
% This template has been downloaded from:
% http://www.LaTeXTemplates.com
%
% Original author:
% Adrien Friggeri (adrien@friggeri.net)
% https://github.com/afriggeri/CV
%
% License:
% CC BY-NC-SA 3.0 (http://creativecommons.org/licenses/by-nc-sa/3.0/)
%
% Important notes:
% This template needs to be compiled with XeLaTeX and the bibliography, if used,
% needs to be compiled with biber rather than bibtex.
%
%%%%%%%%%%%%%%%%%%%%%%%%%%%%%%%%%%%%%%%%%
\RequirePackage{fixltx2e}
\documentclass[]{friggeri-cv} % Add 'print' as an option into the square bracket to remove colors from this template for printing
\usepackage{rotating}
%\addbibresource{bibliography.bib} % Specify the bibliography file to include publications

\begin{document}


\header{rui}{Oliveira}{Software Engineer} % Your name and current job title/field

%----------------------------------------------------------------------------------------
%	SIDEBAR SECTION
%----------------------------------------------------------------------------------------

\begin{aside} % In the aside, each new line forces a line break
\section{info}
Rui Pedro A. Oliveira
Portuguese, UE
1994.11.10
Male
\section{Contacto}
Rua São Martinho nº55
4765-343 V. N. Famalicão
Portugal
~
+(351) 915015377
~
\href{mailto:rui96pedro@gmail.com}{rui96pedro@gmail.com}
~
\href{https://bitbucket.com/ruioliveiras}{Portfolio}
~
\section{Linguas}
Portugues
Inglês 
\section{Highlights}
UML , 
Java, Scala, Php, 
Javascript, AngularJs,
Sql, No-Sql,
Git, Scrum, Jira, 
Linux %{\color{red} $\varheartsuit$}
Asp.net, VisualStudio
~
\end{aside}

%---------------------------------------------------------------------------------------
%			Resumo
%---------------------------------------------------------------------------------------
\section{resumo}
Licenciado em Engenharia Informática na Universidade do Minho (Braga, Portugal), após a realização de um curso profissional de programação. Durante a licenciatura realizei vários trabalhos práticos exigentes, alguns deles detalhados a baixo no portefólio. Também se destaca o trabalho pratico final do curso profissional, que foi premiado num concurso municipal.\\ 
De forma a completar a experiencia académica realizei um estagio de verão no final do 2º ano da licenciatura, e após a conclusão da licenciatura iniciei um projecto financiado por o grupo AdClick.\\ 
Estou sempre pronto para aprender, tenho espírito crítico, sou pró-activo e adapto-me facilmente.

%Student of BSc in Informatics Engineering (some practical works on my Portfolio page)\\ High school: Professional course in programming (internship and final project).
%---------------------------------------------------------------------------------------
%			EDUCATION SECTION
%---------------------------------------------------------------------------------------
\section{educação}

\begin{entrylist}
% LICENCIATURA------------------------------------------
\entry
{2012--2015}
{Licenciatura em Engenharia Informatica}
{Universidade do Minho, Portugal}
{Curso prestigiado na área da informática. Curso abrangente onde se destaca: programação funcional e orientada a objectos; Algoritmos e estruturas de dados; Criação e manutenção de Bases de dados relacionais; Sistemas distribuídos; Multithreading; Modelação de software; Média final: 16.}

%CURSO PROFISSIONAL------------------------------------------
\entry
{2009--2012}
{Curso Profissional {\normalfont Escola secundária}}
{Didaxis - Riba d'ave, Portugal}
{\emph{Curso Profissional de Gestão e Programação de Sistemas da informação}. Destaques: Instrução às linguagens C, .Net e Java; Projecto final: aplicação em Android (ver Prémios); Estagio profissional na Codevison.}

%------------------------------------------------
\end{entrylist}

%---------------------------------------------------------------------------------------
%	WORK EXPERIENCE SECTION
%---------------------------------------------------------------------------------------
\section{experiência}

\begin{entrylist}
%------------------------------------------------
\entry
{2015--2016}
{Grupo Adclick}
{Porto, Portugal}
{\emph{Projecto } \\
Projecto que começou por R\&D, e onde foi desenvolvido um protótipo do sistema em Scala. 
}

\entry
{2014}
{Wondeotec - {\normalfont emailbidding.com}}
{Porto,Portugal}
{\emph{Estagio de verão (2 Meses)} \\
Participação no desenvolvimento do produto EmailBidding. Destaques: Trabalho em equipa, Scrum, JIRA, Php, Synfony. }

%\entry
%{2012}
%{Codevision}
%{Braga,Portugal}
%{\emph{Estagio Profissional (2 months)} \\
%Desenvolvimento de uma aplicação web com Javascript - Sencha Touch.
%}
\end{entrylist}

%-------------------------------------------------------------------------------------
%	AWARDS SECTION
%-------------------------------------------------------------------------------------
\section{prêmios}

\begin{entrylist}
%------------------------------------------------
\entry
{2013}
{O Meu projeto é empreendedor}
{Premio Municipal, Vila Nova de Famalicão (Portugal)}
{Projecto final do Curso Profissional: Uma aplicação Android para gerir o guarda-roupa.}

%------------------------------------------------
\entry
{2012}
{TECLA (1\textsuperscript{st} place)}
{Competição de programação, Aveiro (Portugal)}
{TECLA é um concurso de programação para equipas de 2 elementos, com o objectivo de resolver problemas algorítmicos.}

%------------------------------------------------
\entry
{2011}
{ONI (5\textsuperscript{th} place)}
{Competição de programação, Porto (Portugal)}
{ONI são as olimpíadas nacionais da informática.}

\end{entrylist}

%----------------------------------------------------------------------------------------
%	COMMUNICATION SKILLS SECTION
%----------------------------------------------------------------------------------------

\section{Comunicação}
\begin{entrylist}
    \entry
    {2014}
    {Orador de 'Introdução a Php'}
    {University of Minho, Portugal}
    {Palestra realizada na Universidade, e introduzia-se num curso com 3 palestras: HTML, Php e Laravel Framework.}
\end{entrylist}

% --------
% New Page
% -------
%\newpage
\section{portfolio {\normalfont \normalsize (\href{https://bitbucket.org/ruioliveiras}{https://bitbucket.org/ruioliveiras})}}
\begin{entrylist}

% SEI
\entry
{2015}
{Concurso de Musica - Sistema Cliente/Servidor }
{Universidade do Minho, Portugal}
{
Aplicação criada em Java. Existem vários clientes e existem vários servidores. Os servidores comunicam entre si e cada cliente tem apenas um servidor.
Destaques técnicos:
\begin{itemize} 
    \item Comunicação entre Cliente servidor com UDP Datagrams.
    \item Comunicação entre servidores com sockets TCP.
    \item Criação de protocolo de fragmentação sobre UDP, com politica de recuperação de erros.
\end{itemize}
}
% habitat application
\entry
{2015}
{Aplicação de gestão para a Organização Habitat Portugal}
{Universidade do Minho, Portugal}
{
Aplicação tinha por fim gerir o processo da Habitat. Foi construída em Java com base de dados Mysql. Detalhes técnicos:
\begin{itemize} 
    \item Estrutura em camadas: Interface, Negocio, Dados .
    \item Base de dados com Triggers para campos deriavados.
    \item Criação de Interface generica para fazer CRUD de qualquer tipo de informação.
    \item Estudo dos casos de uso, e otimização da Interface.
\end{itemize}
}

% SEI
\entry
{2015}
{Web Site Semana da Engenharia Informatica (SEI) 2015 \\  {\normalfont \normalsize (\href{https://github.com/cesium/SEI15}{https://github.com/cesium/SEI15})}}
{Universidade do Minho}
{Participação no desenvolvimento do site da SEI 2015. Introduzi AngularJs para evitar repetições de HTML e ser mais fácil de gerir o conteúdo.}

% distributed systems:
\entry
{2014}
{Sistema distribuido \"Warehouse\" Cliente/Servidor }
{Universidade do Minho}
{Cliente/Servidor contruido em Java com socket TCP. Detalhes técnicos:
\begin{itemize} 
    \item Controlo de concurrencia utilizando Locks.
    \item Serialização e descerialização de Objetos para texto.
    \item Implementação manual de metodos RPC sobre o Socket com apenas Texto.
\end{itemize}
}

% Dpum
\entry
{2014}
{Dpum CESIUM code validator}
{Universidade do Minho}
{Criação de Web Site para gerir concurso de programação, o site tinha que executar uma script que compilava o codigo subemito e testava os outputs do programa. O foi criado utilizando PHP-Slim e AngularJs.}

%------------------------------------------------

%\begin{itemize} 
%	\item Login, using session token with time out;
%	\item Manage Items, Users, purchases;
%	\item Add pictures to items;
%\end{itemize}

%------------------------------------------------
\entry
{2013}
{Compressor de ficheiros com o algoritmo de Shannon Fano em C++}
{University of Minho, Portugal}
{Detalhes técnicos: operações bitwise, grafos e C++.
}
%------------------------------------------------
\end{entrylist}


\end{document}
