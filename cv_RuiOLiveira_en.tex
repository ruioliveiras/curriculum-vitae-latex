%%%%%%%%%%%%%%%%%%%%%%%%%%%%%%%%%%%%%%%%%
% Friggeri Resume/CV
% XeLaTeX Template
% Version 1.0 (5/5/13)
%
% This template has been downloaded from:
% http://www.LaTeXTemplates.com
%
% Original author:
% Adrien Friggeri (adrien@friggeri.net)
% https://github.com/afriggeri/CV
%
% License:
% CC BY-NC-SA 3.0 (http://creativecommons.org/licenses/by-nc-sa/3.0/)
%
% Important notes:
% This template needs to be compiled with XeLaTeX and the bibliography, if used,
% needs to be compiled with biber rather than bibtex.
%
%%%%%%%%%%%%%%%%%%%%%%%%%%%%%%%%%%%%%%%%%

\documentclass[]{friggeri-cv} % Add 'print' as an option into the square bracket to remove colors from this template for printing
\usepackage{rotating}
\addbibresource{bibliography.bib} % Specify the bibliography file to include publications

\begin{document}


\header{rui}{Oliveira}{programmer} % Your name and current job title/field

%----------------------------------------------------------------------------------------
%	SIDEBAR SECTION
%----------------------------------------------------------------------------------------

\begin{aside} % In the aside, each new line forces a line break
\section{info}
Rui Pedro A. Oliveira
Portuguese, UE
1994.11.10
Male
\section{contact}
Rua São Martinho nº55
4765-343 V. N. Famalicão
Portugal
~
+(351) 915015377
~
\href{mailto:rui96pedro@hotmail.com}{rui96pedro@hotmail.com}
~
\href{https://bitbucket.org/ruiOliveiras94}{Portfolio}
~
\section{languages}
portuguese
english 
\section{programming}
Java, C, MySql,
Php, Javascript, AngularJs,
Git , Linux {\color{red} $\varheartsuit$} 
~
\end{aside}

%---------------------------------------------------------------------------------------
%			Resumo
%---------------------------------------------------------------------------------------
\section{aboutMe}
Currently attending the 3nd year of the BSc in Informatics Engineering at Minho University (Braga, Portugal), after concluding a high school professional course in programming. Throughout the BSc several practical assignments are delivered to students, some of them detailed in the portfolio below. The high school final project is also highlight. In addition to a good educational background, a short term summer internship was held at the end of the 2nd BSc year.

%Student of BSc in Informatics Engineering (some practical works on my Portfolio page)\\ High school: Professional course in programming (internship and final project).
%---------------------------------------------------------------------------------------
%			EDUCATION SECTION
%---------------------------------------------------------------------------------------
\section{education}

\begin{entrylist}
% LICENCIATURA------------------------------------------
\entry
{2012--2015}
{BSc in Informatics Engineering  {\normalfont (on going) }}
{University of Minho, Portugal}
{Comprehensive and prestige undergraduate course in computational science and engineering. Highlights: object oriented, functional and imperative programming; Algorithms and data structures; Relational database architecture and maintenance; Distributed Systems - concurrency and synchronization. Current degree grade: 16 }
%CURSO PROFISSIONAL------------------------------------------
\entry
{2009--2012}
{Professional Course {\normalfont High School – grade 9 to 12}}
{Didxis, Portugal}
{\emph{Professional Course in Management and Programming of Computer Systems}. Highlights: introduction to C, .Net and Java programming languages. Final project: development of an Android Application (see below in awards). Internship: Codevision (see below in experience).}

%------------------------------------------------
\end{entrylist}

%---------------------------------------------------------------------------------------
%	WORK EXPERIENCE SECTION
%---------------------------------------------------------------------------------------
\section{experience}

\begin{entrylist}
%------------------------------------------------
\entry
{2014}
{Wondeotec}
{Porto,Portugal}
{\emph{Summer Internship (2 months)} \\
Participation in the development of EmailBidding, a email markting platform, made in Symfony2. Highlights: Php, Symfony2, team working with scrum.
}

\entry
{2012}
{Codevision}
{Braga,Portugal}
{\emph{Professional Internship (2 months)} \\
Development of a mobile web application (Javascript - Sencha Touch framework, Visual Studio, MVC Architecture) 
}
\end{entrylist}

%-------------------------------------------------------------------------------------
%	AWARDS SECTION
%-------------------------------------------------------------------------------------
\section{awards}

\begin{entrylist}
%------------------------------------------------
\entry
{2013}
{My Project is Entrepreneur}
{Municipal award, Vila Nova de Famalicão (Portugal)}
{Final project of the Professional Course. Was a Android Application to manage people wardrobe.}
%------------------------------------------------
\entry
{2012}
{TECLA (1\textsuperscript{st} place)}
{Programming Competition, Aveiro (Portugal)}
{TECLA is a programming competition for teams of two elements. The objective is solve algorithmic problems.}
%------------------------------------------------
\entry
{2011}
{ONI (5\textsuperscript{th} place)}
{Programming Competition, Porto (Portugal)}
{ONI is the Portuguese National Olympiads of Informatics competition.}

\end{entrylist}

%----------------------------------------------------------------------------------------
%	COMMUNICATION SKILLS SECTION
%----------------------------------------------------------------------------------------

\section{speaking}
\begin{entrylist}
    \entry
    {2014}
    {Php Introduction}
    {University of Minho, Portugal}
    {This was a curse with 3 Workshop: HTML, Php and Laravel Framework. I Was the speaker of Php Introduction Workshop.}
\end{entrylist}

% --------
% New Page
% -------
\newpage
\section{portfolio {\normalfont \normalsize (\href{https://bitbucket.org/ruiOliveiras94}{https://bitbucket.org/ruiOliveiras94})}}
\begin{entrylist}

% SEI
\entry
{2014}
{SEI 2015 web  {\normalfont \normalsize (\href{https://github.com/cesium/SEI15}{https://github.com/cesium/SEI15})}}
{University of Minho, Portugal}
{Participation in the development of SEI 2015 Web-Site, I was introduced angularJS to avoid html repetitions.}

% distributed systems:
\entry
{2014}
{Distributed System \"Warehouse\" Client/Server }
{University of Minho, Portugal}
{Client/Server made in Java using socket, technical highlights:
\begin{itemize} 
    \item Concurrency control with reentrant locks and conditions.
    \item Communication inside socked with pure text using serializing and deserializing.
\end{itemize}
}
% habitat application
\entry
{2014}
{CRUD management Application Habitat Portugal}
{University of Minho, Portugal}
{CRUD management Application made in Java with Mysql database with 22 tables, some technical highlights:
\begin{itemize} 
    \item Three layers: View, Business, Data.
    \item Database with triggers to derivative attributes
    \item Generic interface management
\end{itemize}
}
% Dpum
\entry
{2014}
{Dpum CESIUM code validator}
{University of Minho, Portugal}
{This web site was created to manage an programing conquest in Minho university.
 This site was made just to one problem at time, and was used in production environment.}


%------------------------------------------------
\entry
{2014}
{Web service REST in PHP with Login System {\normalfont(in progress)}}
{University of Minho, Portugal}
{Web service about a Warehouse. Features: Login using session token with time out; manage items, users, purchases; add pictures to items;

%\begin{itemize} 
%	\item Login, using session token with time out;
%	\item Manage Items, Users, purchases;
%	\item Add pictures to items;
%\end{itemize}
}

%------------------------------------------------
\entry
{2014}
{File Compressor in C++, Shannon Fano algorithm}
{University of Minho, Portugal}
{Implementation highlights: Graphs algorithms and Bitwise operation.
%\begin{itemize} 
%	\item Graphs algorithms;
%	\item Dynamic allocated memory;
%	\item Bitwise operation;
%\end{itemize}
}
%------------------------------------------------
\end{entrylist}


\end{document}
