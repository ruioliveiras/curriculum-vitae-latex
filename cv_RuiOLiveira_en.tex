%%%%%%%%%%%%%%%%%%%%%%%%%%%%%%%%%%%%%%%%%
% Friggeri Resume/CV
% XeLaTeX Template
% Version 1.0 (5/5/13)
%
% This template has been downloaded from:
% http://www.LaTeXTemplates.com
%
% Original author:
% Adrien Friggeri (adrien@friggeri.net)
% https://github.com/afriggeri/CV
%
% License:
% CC BY-NC-SA 3.0 (http://creativecommons.org/licenses/by-nc-sa/3.0/)
%
% Important notes:
% This template needs to be compiled with XeLaTeX and the bibliography, if used,
% needs to be compiled with biber rather than bibtex.
%
%%%%%%%%%%%%%%%%%%%%%%%%%%%%%%%%%%%%%%%%%
\RequirePackage{fixltx2e}
\documentclass[]{friggeri-cv} % Add 'print' as an option into the square bracket to remove colors from this template for printing
\usepackage{rotating}
\addbibresource{bibliography.bib} % Specify the bibliography file to include publications

\begin{document}


\header{rui}{Oliveira}{programmer} % Your name and current job title/field

%----------------------------------------------------------------------------------------
%	SIDEBAR SECTION
%----------------------------------------------------------------------------------------

\begin{aside} % In the aside, each new line forces a line break
\section{info}
Rui Pedro A. Oliveira
Portuguese, UE
1994.11.10
Male
\section{contact}
Rua São Martinho nº55
4765-343 V. N. Famalicão
Portugal
~
+(351) 915015377
~
\href{mailto:rui96pedro@gmail.com}{rui96pedro@gmail.com}
~
\href{https://bitbucket.org/ruiOliveiras94}{Portfolio}
~
\section{languages}
portuguese
english 
\section{programming}
Java, C, MySql,
Php, Javascript, AngularJs,
Git , Linux {\color{red} $\varheartsuit$} 
Asp.net, VisualStudio
~
\end{aside}

%---------------------------------------------------------------------------------------
%			Resumo
%---------------------------------------------------------------------------------------
\section{aboutMe}
BSc in Informatics Engineering at Minho University (Braga, Portugal)
Currently attending the 3rd year of the BSc in Informatics Engineering at Minho University (Braga, Portugal), after concluding a high school professional programming course. Throughout the BSc, the students are required to do several practical assignments, being some of them detailed in the portfolio below. The high school final project is also highlighted. In addition to a good educational background, a short term summer internship was held at the end of the 2nd year of study.

%Student of BSc in Informatics Engineering (some practical works on my Portfolio page)\\ High school: Professional course in programming (internship and final project).
%---------------------------------------------------------------------------------------
%			EDUCATION SECTION
%---------------------------------------------------------------------------------------
\section{education}

\begin{entrylist}
% LICENCIATURA------------------------------------------
\entry
{2012--2015}
{BSc in Informatics Engineering }
{University of Minho, Portugal}
{Comprehensive and prestiged undergraduate course in computer science and informatics engineering. Highlights: object oriented, functional and imperative programming; Algorithms and data structures; Relational database architecture and management; Distributed Systems - concurrency and synchronization. Current degree grade: 16 }
%CURSO PROFISSIONAL------------------------------------------
\entry
{2009--2012}
{Professional Course {\normalfont High School – grade 9 to 12}}
{Didaxis, Portugal}
{\emph{Professional Course in Management and Programming of Computer Systems}. Highlights: introduction to C, .Net and Java programming languages. Final project: development of an Android Application (see below in awards). Internship: Codevision (see below in experience).}

%------------------------------------------------
\end{entrylist}

%---------------------------------------------------------------------------------------
%	WORK EXPERIENCE SECTION
%---------------------------------------------------------------------------------------
\section{experience}

\begin{entrylist}
%------------------------------------------------
\entry
{2014}
{Wondeotec - {\normalfont emailbidding.com}}
{Porto,Portugal}
{\emph{Summer Internship (2 months)} \\
Participation in the development of EmailBidding, a email marketing platform, made in Symfony2. Highlights: Php, Symfony2, team working with scrum and JIRA.
}

\entry
{2012}
{Codevision}
{Braga,Portugal}
{\emph{Professional Internship (2 months)} \\
Development of a mobile web application (Javascript - Sencha Touch framework, Visual Studio, MVC Architecture) 
}
\end{entrylist}

%-------------------------------------------------------------------------------------
%	AWARDS SECTION
%-------------------------------------------------------------------------------------
\section{awards}

\begin{entrylist}
%------------------------------------------------
\entry
{2013}
{My Project is Entrepreneur}
{Municipal award, Vila Nova de Famalicão (Portugal)}
{Final project of the Professional Course. Android Application to manage people wardrobe.}
%------------------------------------------------
\entry
{2012}
{TECLA (1\textsuperscript{st} place)}
{Programming Competition, Aveiro (Portugal)}
{TECLA is a programming competition for teams of two elements. The objective is to solve algorithmic challenges.}
%------------------------------------------------
\entry
{2011}
{ONI (5\textsuperscript{th} place)}
{Programming Competition, Porto (Portugal)}
{ONI is the Portuguese National Olympiads of Informatics.}

\end{entrylist}

%----------------------------------------------------------------------------------------
%	COMMUNICATION SKILLS SECTION
%----------------------------------------------------------------------------------------

\section{speaking}
\begin{entrylist}
    \entry
    {2014}
    {Php Introduction}
    {University of Minho, Portugal}
    {Workshop based course. Subjects: HTML, Php and Laravel Framework.}
\end{entrylist}

% --------
% New Page
% -------
\newpage
\section{portfolio {\normalfont \normalsize (\href{https://bitbucket.org/ruioliveiras}{https://bitbucket.org/ruioliveiras})}}
\begin{entrylist}

% SEI
\entry
{2015}
{Musical challenge application Client/Server}
{University of Minho, Portugal}
{
    Application created with Java. Allows the existence of many inter-connected servers, with each client connecting to one server. Technical highlights:
\begin{itemize} 
    \item Communication between the Client and the Server using UDP Datagrams
    \item Communication between Servers using TCP sockets
    \item UPD datagram fragmentation system
\end{itemize}
}


% SEI
\entry
{2014}
{SEI 2015 web  {\normalfont \normalsize (\href{https://github.com/cesium/SEI15}{https://github.com/cesium/SEI15})}}
{University of Minho, Portugal}
{Participation in the development of SEI 2015 Web-Site. Implemented angularJS usage to avoid html repetitions.}

% distributed systems:
\entry
{2014}
{Distributed System \"Warehouse\" Client/Server }
{University of Minho, Portugal}
{Client/Server made with Java using socket. Technical highlights:
\begin{itemize} 
    \item Concurrency control with reentrant locks and conditions.
    \item Communication inside sockets with pure text using serializing and deserializing.
\end{itemize}
}
% habitat application
\entry
{2014}
{CRUD management Application for Habitat Portugal}
{University of Minho, Portugal}
{CRUD management Application made with Java and Mysql database. Technical highlights:
\begin{itemize} 
    \item Three layers: View, Business, Data.
    \item Database with triggers to derivative attributes
    \item Generic interface design
\end{itemize}
}
% Dpum
\entry
{2014}
{Dpum CESIUM code validator}
{University of Minho, Portugal}
{Web site created to manage a programming conquest at Minho university.}

%------------------------------------------------

%\begin{itemize} 
%	\item Login, using session token with time out;
%	\item Manage Items, Users, purchases;
%	\item Add pictures to items;
%\end{itemize}

%------------------------------------------------
\entry
{2013}
{File Compressor with C++, Shannon Fano algorithm}
{University of Minho, Portugal}
{Implementation highlights: Graph based algorithms and Bitwise operations.
%\begin{itemize} 
%	\item Graphs algorithms;
%	\item Dynamic allocated memory;
%	\item Bitwise operation;
%\end{itemize}
}
%------------------------------------------------
\end{entrylist}


\end{document}
